\subsection{Classification}
    \begin{lstlisting}
        # Régression logistique
        glm(formule, data = data, family = binomial)
        # LDA
        lda(formule, data = data)
        # QDA
        qda(formule, data = data)
        # k-NN (attention, il faut avoir train.X, train.y et test.X)
        knn(train.X, test.X, train.y, k = 1)
        # Arbres
        tree(formule, data = data)
        # Forêts aléatoires en utilisant m variables et n arbres
        randomForest(formule, data = data, mtry = m, ntrees = n)
    \end{lstlisting}

    \subsubsection{Récupérer probabilités}
        \lstinline{predict} supporte le paramètre \lstinline{type = "prob"} pour récupérer les probabilités pour certains modèles. Pour d'autres (comme \lstinline{glm}), il faut \lstinline{type = "response"}.