\subsection{Divers}
    \subsubsection{Opérations matricielles}
        Pour multiplier deux matrices \(A\) et \(B\), il faut écrire \lstinline{A %*% B}.

    \subsubsection{Génération aléatoire}
        \begin{lstlisting}
            # Contrôle la graine
            set.seed(1)
            # Générer un vecteur de taille n
            # selon une loi normale de moyenne m et d'écart-type s
            rnorm(n, mean = m, sd = s)
            # selon une loi uniforme entre a et b
            runif(n, min = a, max = b)
            # selon une loi de Poisson avec comme paramètre l
            rpois(n, lambda = l)
        \end{lstlisting}

    \subsubsection{Tidyverse}
        \begin{lstlisting}
            dataset = dataset %>%
                # Ajoute ou transforme une colonne
                mutate(column = transformation(other.column)) %>%
                # Filtre les données (par exemple, filter(!is.na(column)))
                filter(condition(column)) %>%
                # Retire les lignes dupliquées
                distinct(column) %>%
                # Sélectionne des colonnes
                select(col1, col2, ...) %>%
                # Groupe les données en ligne avec la même valeur pour column
                group_by(column) %>%
                # Retire les groupes
                ungroup() %>%
                # Rassemble plusieurs colonnes en une paires (clé, valeurs)
                # Les autres colonnes sont dupliquées
                gather(col1, col2, col3, ..., key = "keycolumn", value = "valuecolumn") %>%
                # Opération inverse (découpe les paires en colonnes)
                spread(key = "keycolumn", value = "valuecolumn")
                # Summarise les données en une seule ligne avec les valeurs
                summarise(new.column = summary.function(column))
        \end{lstlisting}

        Fonctions supportées par \lstinline{summarise} :
        \begin{itemize}
            \item \lstinline{first}, \lstinline{last}, \lstinline{nth} pour récupérer les valeurs
            \item \lstinline{n} pour le nombre de valeurs, \lstinline{n_distinct} pour le nombre de valeurs différentes
            \item \lstinline{min}, \lstinline{max}, \lstinline{mean}, \lstinline{median}, \lstinline{var}, \lstinline{var} pour récupérer les valeurs statistiques associées
        \end{itemize}

    \subsubsection{Couper un jeu de données}
        On va couper aléatoirement un jeu de données en un ensemble d'entraînement (avec 2000 données) et un ensemble de test (avec toutes les autres données).

        \begin{lstlisting}
            train = sample(1:nrow(data), 2000)
            data.train = data[train,]
            data.test = data[-train,]
        \end{lstlisting}