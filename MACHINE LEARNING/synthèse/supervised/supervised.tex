\section{Apprentissage supervisé}
    \subsection{Éléments communs}
        \begin{definition}
            Les \textit{variables d'entrée}\index{Variable!d'entrée} (\textit{input variables}\index{Input variable|see {Variable, d'entrée}}), aussi appelées \textit{predictors}\index{Predictor|see {Variable, d'entrée}}, \textit{variables indépendantes}\index{Variable!indépendante|see {Variable, d'entrée}}, \textit{features}\index{Feature|see {Variable, d'entrée}} ou simplement \textit{variables}, sont représentées par le symbole \(X\). S'il y a \(p\) variables différentes, on écrit :
            \[X = (X_1, X_2, \dots, X_p)\]
        \end{definition}

        \begin{remarque}
            \(X\) est une variable aléatoire. Une réalisation de cette variable (comme, par exemple, une instance dans un jeu de données) est notée \(x\).
        \end{remarque}

        \begin{definition}
            La \textit{variable de sortie}\index{Variable!de sortie} (\textit{output variable}\index{Output variable|see {Variable, de sortie}}), aussi appelée \textit{réponse}\index{Réponse|see {Variable, de sortie}} ou \textit{variable dépendante}\index{Variable!dépendante|see {Variable, de sortie}} est souvent notée \(Y\).
        \end{definition}

        On suppose qu'il existe une relation aléatoire entre \(X\) et \(Y\), c'est-à-dire :
        \[\distribution(X, Y) = \distribution(X)\distribution(Y|X) \not= \distribution(X)\distribution(Y)\]

        \begin{definition}
            Les \textit{données}\index{Données} :
            \[
                \data = \{(x_1, y_1), (x_2, y_2), \dots, (x_n, y_n)\} = \{(x_i, y_i\}^n_{i=1}
            \]
            où \(x_i = (x_{i,1}, \dots, x_{i,p})\), c'est-à-dire que \(x_i\) est le tuple des variables pour la i\ieme{} instance (ligne) dans les données.

            On peut voir les données comme un échantillon i.i.d. de la véritable distribution, c'est-à-dire :
            \[(x_i, y_i) \stackrel{i.i.d.}{\sim} \distribution(X, Y)\]
            \nomenclature{i.i.d.}{Indépendant et identiquement distribué}
        \end{definition}

        On peut décrire tous les problèmes d'apprentissage supervisé avec les éléments suivants :
        \begin{itemize}
            \item Les \textit{variables d'entrée}\index{Variable!d'entrée} \(X\) et la \textit{variable de sortie}\index{Variable!de sortie} \(Y\);
            \item Les \textit{données}\index{Données} \(\data = \{(x_i, y_i)_{i=1}^n\}\);
            \item La \textit{fonction cible}\index{Fonction!cible}. Par exemple, la distribution jointe \(\distribution(X, Y)\), l'espérance conditionnelle \(\expectation[Y|X]\), la distribution conditionnelle \(\distribution(Y|X)\), etc.;
            \item L'\textit{ensemble d'hypothèses}\index{Ensembles d'hypothèses} \(\hypo\) contient toutes les hypothèses à considérer (les fonctions à tester). Souvent, cet ensemble est défini implicitement; et
            \item La \textit{fonction de perte}\index{Fonction!de perte} (ou \textit{fonction de coût}\index{Fonction!de coût|see {Fonction, de perte}}) \(L(y, \estimation{y})\) est une fonction \(\R \times \R \to \R^+\). Par exemple, \(L(y, \estimation{y}) = (y - \estimation{y})^2\).
        \end{itemize}
        L'\textit{algorithme d'apprentissage}\index{Algorithme!d'apprentissage} choisi la meilleure hypothèse de \(\hypo\) en utilisant les données \(\data\) et la fonction de perte \(L\).

        \subsubsection{Erreurs}
            \begin{definition}
                L'\textit{erreur de test}\index{Erreur!de test} (ou \textit{erreur out-of-sample}\index{Erreur!out-of-sample|see {Erreur, de test}}) de l'hypothèse \(\fEstimated\) permet de tester la performance de l'hypothèse sur de nouvelles données et est défini par :
                \[
                    \errorOut(\fEstimated) = \expectation[L(Y, \fEstimated(X))]
                \]
            \end{definition}

            On veut sélectionner la meilleure hypothèse :
            \[
                \fEstimated = \argmin_{h \in \hypo} \errorOut(\widehat{h}_\data)
            \]

            Cependant, on ne sait pas calculer \(\errorOut(\fEstimated)\) car on ne connait pas \(\distribution(Y, X)\).

        \subsubsection{Biais et variance}
            \begin{definition}
                Le \textit{biais}\index{Biais} est l'erreur introduite en modélisant un problème compliqué par un problème plus simple.

                La \textit{variance}\index{Variance} permet de mesurer à quel point le modèle construit serait différent si les données d'entraînement étaient différentes.
            \end{definition}

            En général, une méthode plus flexible implique un biais plus petit et une variance plus grande (car la méthode colle mieux aux données d'entraînement).

            \begin{remarque}
                La taille de l'ensemble d'entraînement a un impact sur la variance. Un plus grand ensemble réduit la variance !
            \end{remarque}

    \section{Régression avancée}
    Rappel : solution optimale de la régression linéaire donnée par OLS : \(\betaLS = (X^TX)^{-1} X^Ty\)

    Cette solution peut avoir des problèmes en haute dimension parce qu'il peut arriver que certaines colonnes de \(X\) ne sont pas linéairement indépendantes. Par conséquent, \(X\) n'est pas `full rank'. Dans ce cas, \(X^T X\) est singulière (non-inversible) et les coefficients OLS n'ont pas une unique valeur.

    Pour rappel, le biais de \(\betaLS\) est \(\expectation[\betaLS] - \beta^* = 0\) et la variance est \(\variance[\betaLS] = \sigma^2 (X^T X)^{-1}\). Si \(X\) a des colonnes orthonormales :
    \[
        \trace(\variance[\betaLS]) = \sigma^2 p
    \]

    On peut réduire le nombre de dimensions en appliquant PCA (dans ce cas, on fait du \textit{principal components regression} ou PCR\index{PCR}). Ceci suppose que les directions dans lesquels \(X_1, \dots, X_p\) ont la plus grande variation ont une relation avec \(Y\).

    Une autre solution est d'utiliser le \textit{best subset selection}\index{Best subset selection} :
    \begin{align*}
        &\min_{\beta} \left\{\sum_{i=1}^n \left(y_i - \beta_0 - \sum_{j=1}^p \beta_j x_{ij}\right)^2\right\}\\
        &\text{sous contrainte } \sum_{j=1}^p \identity(\beta_j \not= 0) \leq s
    \end{align*}
    avec \(s \geq 0\) un hyper-paramètre. Il faut considérer \(p \choose s\) modèles contenant \(s\) prédicteurs. Ceci est infaisable quand \(p\) et \(s\) sont grands. De plus, quand l'espace de recherche est grand, il y a de grands risques d'overfitting. On peut utiliser les stepwise procedures pour contrecarrer ça (mais ces méthodes ne garantissent pas d'obtenir l'optimal).

    \subsection{Shrinkage}
        Finalement, la dernière solution du cours est le \textit{shrinkage}\index{Shrinkage}.
        \begin{definition}
            Le \textit{shrinkage} (ou \textit{regularization}\index{Regularization|see {Shrinkage}}) consiste à fitter un modèle utilisant les \(p\) variables mais les coefficients estimés sont réduits vers zéro par rapport aux coefficients de OLS. Ceci a pour effet de réduire la variance et de faire de la sélection de variables.
        \end{definition}

        \subsubsection{Ridge regression}
            \begin{definition}
                La \textit{régression Ridge}\index{Ridge} est le problème d'optimisation suivant :
                \begin{align*}
                    &\min_{\beta} \left\{\sum_{i=1}^n \left(y_i - \beta_0 - \sum_{j=1}^p \beta_j x_{ij}\right)^2\right\}\\
                    &\text{sous contrainte } \sum_{j=1}^p \beta_j^2 \leq s
                \end{align*}
                avec \(s \leq 0\) un hyper-paramètre.

                Si \(s = 0\), \(\betaRidge = (0, \dots, 0)\). Si \(s = +\infty, \betaRidge = \betaLS\). Si \(s \in ]0, +\infty[\), on fait du tradeoff.
            \end{definition}

            Géométriquement, la contrainte de régularisation est une sphère.

            \begin{definition}
                On peut formuler la régression ridge comme :
                \[
                    \min_{\beta} \sum_{i=1}^n \left(y_i - \beta_0 - \sum_{j=1}^p \beta_j x_{ij}\right)^2 + \lambda \sum_{j=1}^p \beta_j^2
                \]
                avec \(\lambda \leq 0\) un hyper-paramètre.

                Si \(\lambda = 0\), \(\betaRidge = \betaLS\). Si \(\lambda = \infty\), \(\betaRidge = (0, \dots, 0)\). Si \(\lambda \in (0, \infty)\), on fait du tradeoff.

                On peut résoudre ce problème par augmentation des données :
                \[
                    \betaRidge = (X^T X + \lambda \identity_p)^{-1} X^T y
                \]
            \end{definition}

            \paragraph{Un cas simple}
                Supposons que \(n = p\) et que \(X = \identity_n = \identity_p\), alors :
                \begin{align*}
                    &\min_\beta \sum_{j=1}^p (y_j - \beta_j)^2 &\implies \betaLS &= y_j\\
                    &\min_\beta \sum_{j=1}^p (y_j - \beta_j)^2 + \lambda\sum_{j=1}^p \beta_j^2 &\implies \betaRidge &= \frac{y_j}{1+\lambda} = \frac{\betaLS}{1 + \lambda}
                \end{align*}

            \paragraph{Scaling}
                Les coefficients OLS sont `scale equivariant' (multiplier \(X_j\) par une constante \(c \implies\) les coefficients OLS sont divisés par \(c\); peu importe comment la \(j\)\ieme{} variable est scalée, \(X_j\betaEstimated{j}\) est toujours le même). En revanche, les coefficients Ridge peuvent changer drastiquement (à cause de la somme des coefficients au carré).

            \paragraph{Biais et variance}
                Posons \(R = X^T X\)
                \begin{align*}
                    \betaRidge &= (X^T X + \lambda \identity_p)^{-1} X^T y = (R + \lambda \identity_p)^{-1} R(R^{-1} X^T y)\\
                    &= (R + \lambda \identity_p)^{-1} R \betaLS = \left[R(\identity_p + \lambda R^{-1})\right]^1 R \betaLS\\
                    &= (\identity_p + \lambda R^{-1})^{-1} \betaLS
                \end{align*}
                
                Posons \(W_\lambda = (\identity_p + \lambda R^{-1})^{-1}\)
                \begin{align*}
                    \expectation[\betaRidge] &= \expectation[W_\lambda \betaLS] = W_\lambda B \not= \beta & \text{Si \(\lambda \not= 0\)}\\
                    \variance[\betaRidge] &= \variance[W_\lambda \betaLS] = W_\lambda \variance[\betaLS] W_\lambda^T \leq \variance[\betaLS]
                \end{align*}
                On a donc un biais mais un plus petite variance.

            \paragraph{Décomposition en valeurs singulières}
                On peut utiliser la décomposition en valeurs singulières\index{SVD}, c'est-à-dire \(X = U D V^T\). Donc,
                \begin{align*}
                    X^T X &= (UDV^T)^T UDV^T\\
                    &= VD^TU^T UDV^T\\
                    &= VD^T DV^T & \text{Car les colonnes de \(U\) sont orthonormales}\\
                    &= V D^2 V^T & \text{Car \(D\) est diagonale}\\
                    \\
                    \betaLS &= (X^T X)^{-1} X^T y & \text{Définition}\\
                    &= V D^{-1} U^T y & X = U D V^T \text{ et \(V^{-1} = V^T\) (car colonnes orthonormales)}\\
                    &= V D^{-2} D U^T y\\
                    \betaRidge &= (X^T X + \lambda \identity_p)^{-1} X^T y & \text{Définition}\\
                    &= (V D^2 V^T + \lambda VV^T)^{-1} VDU^T y & VV^T = \identity_p \text{ car colonnes orthonormales}\\
                    &= (V(D^2 + \lambda \identity_p) V^T)^{-1} V D U^T y\\
                    &= V(D^2 + \lambda \identity_p)^{-1} V^T V D U^T y\\
                    &= V(D^2 + \lambda \identity_p)^{-1} D U^T y
                \end{align*}
                Les deux expressions ont quasiment la même formulation à une différence près : \(+ \lambda\) pour Ridge. On peut aussi déterminer l'expression des \(\estimation{y}\) :

                \begin{align*}
                    \yLS &= X \betaLS\\
                    &= U D V^T V D^{-2} D U^T y\\
                    &= U D D^{-2} D U^T y\\
                    &= U U^T y\\
                    \yRidge &= X\betaRidge\\
                    &= X V (D^2 + \lambda \identity_p)^{-1} D U^T y\\
                    &= U D V^T V (D^2 + \lambda \identity_p)^{-1} D U^T y\\
                    &= U D (D^2 + \lambda \identity_p)^{-1} D U^T y\\
                    &= U \diagonal(\frac{d_j^2}{d_j^2 + \lambda}) U^T y & \text{\(D\) est diagonale}
                \end{align*}

                Comme \(\lambda \geq 0\), on a \(\frac{d_j^2}{d_j^2 + \lambda} \leq 1\). On a donc bien un effet de shrinkage. Quand \(d_j^2\) est petit, le shrinkage est plus important. La variable \(z_j = X v_j = u_j d_j\) est la \(j\)\ieme{} PC de \(X\).

        \subsubsection{Lasso}
            \begin{definition}
                La \textit{régression Lasso}\index{Lasso}\nomenclature{Lasso}{Least absolute shrinkage and selection operator} est le problème d'optimisation :
                \begin{align*}
                    &\min_\beta \left\{\sum_{i=1}^n \left(y_i - \beta_0 - \sum_{j=1}^p \beta_j x_{ij}\right)^2\right\}\\
                    &\text{sous contrainte} \sum_{j=1}^p |\beta_j| \leq s
                \end{align*}
                avec \(s \geq 0\) un hyper-paramètre.

                Notons \(\betaLasso\) l'estimation des coefficients pour la régression Lasso.

                Si \(s = 0, \betaLasso = (0, \dots, 0)\). Si \(s = \infty, \betaLasso = \betaLS\). Si \(s \in ]0, \infty[\), on fait du tradeoff.
            \end{definition}

            \begin{definition}
                De manière équivalente, on peut exprimer la régression Lasso comme :
                \[
                    \min_\beta \sum_{i=1}^n \left(y_i - \beta_0 - \sum_{j=1}^p \beta_j x_{ij}\right)^2 + \lambda \sum_{j=1}^p |\beta_j|
                \]
                où \(\lambda \geq 0\) est un hyper-paramètre.

                Si \(s = 0, \betaLasso = \betaLasso\). Si \(s = \infty, \betaLasso = (0, \dots, 0)\). Si \(s \in ]0, \infty[\), on fait du tradeoff.
            \end{definition}

            \paragraph{Un cas simple}
                Supposons que \(n = p\) et que \(X = \identity_p = \identity_n\). Dans ce cas, la régression Lasso est 
                \[
                     \min_\beta \sum_{j=1}^p (y_j - \beta_j)^2 + \lambda \sum_{j=1}^p |\beta_j|
                \]
                Alors, on a :
                \[
                    \betaLasso_j = \begin{cases}
                        y_j - \frac{\lambda}{2} & \text{si } y_j > \frac{\lambda}{2}\\
                        y_j + \frac{\lambda}{2} & \text{si } y_j < \frac{-\lambda}{2}\\
                        0                       & \text{si } |y_j| \leq \frac{\lambda}{2}
                    \end{cases}
                \]

                La régression Lasso shrink les coefficients OLS vers zéro par la même constante \(\frac{\lambda}{2}\) et entièrement vers zéro quand ces coefficients sont plus petits que \(\frac{\lambda}{2}\) en valeur absolue.
    \subsection{Classification}
    \begin{lstlisting}
        # Régression logistique
        glm(formule, data = data, family = binomial)
        # LDA
        lda(formule, data = data)
        # QDA
        qda(formule, data = data)
        # k-NN (attention, il faut avoir train.X, train.y et test.X)
        knn(train.X, test.X, train.y, k = 1)
        # Arbres
        tree(formule, data = data)
        # Forêts aléatoires en utilisant m variables et n arbres
        randomForest(formule, data = data, mtry = m, ntrees = n)
    \end{lstlisting}

    \subsubsection{Récupérer probabilités}
        \lstinline{predict} supporte le paramètre \lstinline{type = "prob"} pour récupérer les probabilités pour certains modèles. Pour d'autres (comme \lstinline{glm}), il faut \lstinline{type = "response"}.