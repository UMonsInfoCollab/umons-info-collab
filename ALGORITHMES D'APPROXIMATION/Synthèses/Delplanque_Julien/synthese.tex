\documentclass[a4paper,11pt]{report}

\usepackage[utf8]{inputenc}
\usepackage[T1]{fontenc}
\usepackage[francais]{babel}
\usepackage{amsfonts}
\usepackage{amsmath}
\usepackage{amsthm}
\usepackage{listings}
\usepackage{fullpage}
\usepackage{graphicx}

\renewcommand{\thesection}{\arabic{section}}
\newcommand{\algo}[1]{\textsc{#1}}

\newtheorem{exercice}{Exercice}
\newtheorem{solution}{Solution}

\title{Algorithmes d'approximation: Retranscription des exercices}
\author{Julien Delplanque}
\begin{document}
\maketitle
\newpage

\section{Note}
Ce document est issus des notes d'exercices de plusieurs étudiants de Master 1
prises lors de l'année académique $2015$ - $2016$.

\section{TP 1}
\begin{exercice}
Les deux graphes n'ont pas été redessiné ici, voir les notes manuscrites...\\

\noindent Écrivez $P_2$ sous la forme primale d'un problème d'optimisation
linéaire en nombres entiers.
\end{exercice}

\begin{solution}
$\min(\sum\limits_j{w_jx_j})$ où $1 \le j \le 12$, $x_i \in \{0,1\} \forall i$
t.q
$$
x_i =
\begin{cases}
1 & \text{si } S_i \in I\\
0 & \text{sinon}
\end{cases}
$$
et $w=(6,10,8,6,5,9,4)$\\
s.l.c\\
$$
\begin{array}{ccccccc}
    &   & x_1 & + & x_4 & \ge & 1\\
    &   & x_2 & + & x_4 & \ge & 1\\
    &   & x_3 & + & x_4 & \ge & 1\\
x_1 & + & x_4 & + & x_5 & \ge & 1\\
x_2 & + & x_4 & + & x_5 & \ge & 1\\
    &   & x_3 & + & x_4 & \ge & 1\\
    &   & x_1 & + & x_5 & \ge & 1\\
x_2 & + & x_5 & + & x_6 & \ge & 1\\
    &   & x_3 & + & x_6 & \ge & 1\\
    &   & x_1 & + & x_7 & \ge & 1\\
x_2 & + & x_6 & + & x_7 & \ge & 1\\
    &   & x_7 & + & x_3 & \ge & 1
\end{array}
$$

$Z_{\text{LP}}^* = 22$ avec
$x^*(\frac{1}{2},0,\frac{1}{2},1,\frac{1}{2},\frac{1}{2},\frac{1}{2})$
\end{solution}

\begin{exercice}
Écrivez sa forme dual et sa relaxation linéaire.
\end{exercice}

\begin{solution}
$\max(\sum\limits_i{y_i})$ où $1 \le i \le 12$ et
$y_i \in \mathbb{N}$ (si relaxé: $y_i \in \mathbb{R}$) $y_i \ge 0, \forall i$\\
s.l.c\\
$$
\begin{array}{ccccccccccccc}
    &   &     &   & y_1 & + & y_4 & + & y_7 & + & y_{10} & \le & 6\\
    &   &     &   & y_2 & + & y_5 & + & y_8 & + & y_{11} & \le & 11\\
    &   &     &   & y_3 & + & y_6 & + & y_8 & + & y_{12} & \le & 8\\
y_1 & + & y_2 & + & y_3 & + & y_4 & + & y_5 & + & y_6    & \le & 6\\
    &   &     &   & y_4 & + & y_5 & + & y_7 & + & y_8    & \le & 5\\
    &   &     &   &     &   & y_8 & + & y_9 & + & y_{11} & \le & 9\\
    &   &     &   &     &   & y_{10}&+& y_{11}&+& y_{12} & \le & 4
\end{array}
$$

\noindent $z_{LP}^* = 22$ avec
$(y^*)^T = (0,6,0,0,0,0,3,2,7,3,0,1)$
\end{solution}

\begin{exercice}
Calculez le paramètre $f$ de l'algorithme \algo{DET-ROUND-SC}.
\end{exercice}

\begin{solution}
$$
\begin{array}{ccl}
f & = & \max\limits_{i\in\{1,\dots,12\}}(f_i)$ où $f_i = |\{j : e_i \in S_j\}|\\
  & = & 3
\end{array}
$$

Donc la solution sera au pire $3$ fois la solution optimale:
$\text{APP} \le 3 \text{OPT}$
\end{solution}

\begin{exercice}
Utilisez \algo{DET-ROUND-SC} pour calculer une solution au problème en nombre
entiers et verifiez la garantie à fortiori de l'algorithme.
\end{exercice}

\begin{solution}
On construit la solution approchée au problème $P_2$ de la manière suivante:
$$
x_i =
\begin{cases}
0 & \text{si } x_i^* < \frac{1}{f}, x^*\text{ étant la solution du
problème relaxé}\\
1 & \text{sinon}
\end{cases}
$$

\noindent Ici, $x=(1,0,1,1,1,1,1)$, $\text{APP}=38$.\\
$\text{OPT}=23$ est atteint en selectionnant $S_3, S_4, S_5$ et $S_7$.
$$\frac{\text{APP}}{\text{OPT}} = \frac{38}{23}$$\\

\noindent La garantie à fortiori est donnée par:
$$
\alpha = \frac{\text{APP}}{Z_{LP}^*} = \frac{38}{22}
$$

\noindent Cela implique que $\text{APP} \le 2\text{OPT}$
\end{solution}

\begin{exercice}
Utilisez \algo{DUAL-ROUND-SC} et vérifiez le facteur d'approximation.
\end{exercice}

\begin{solution}
Injectons les $y^*_i$ de $(y^*)^T = (0,6,0,0,0,0,3,2,7,3,0,1)$ dans les
équations du dual de la solution de l'exercice 2.\\

\noindent On va sélectionner $S_j$ dans la solution approchée ssi
$\sum\limits_{i \text{ t.q } v_i \in S_j}{y_i} = w_j$
$$
\begin{array}{ccccccccccccc}
    &   &     &   & 0 & + & 0 & + & 3 & + & 3 & =   & 6\\
    &   &     &   & 6 & + & 0 & + & 3 & + & 0 & \le & 11\\
    &   &     &   & 0 & + & 0 & + & 7 & + & 1 & =   & 8\\
 0  & + &  6  & + & 0 & + & 0 & + & 0 & + & 0 & =   & 6\\
    &   &     &   & 0 & + & 0 & + & 3 & + & 2 & =   & 5\\
    &   &     &   &     &   & 2 & + & 7 & + & 0 & = & 9\\
    &   &     &   &     &   & 3 & + & 0 & + & 1 & = & 4
\end{array}
$$

\noindent On obtient la même solution que pour \algo{DET-ROUND-SC}:
$$
y=(1,0,1,1,1,1,1)
$$

On a bien $38 \le 23 f = 23 . 3 = 69$.
\end{solution}

\begin{exercice}
Utilisez \algo{GREEDY-SC} pour résoudre $P_1$.
\end{exercice}

\begin{solution}
\noindent\textbf{Données:}\\
\begin{center}
\begin{tabular}{lr}
$
\begin{array}{c|c}
i & S_i\\
\hline
1 & \{1,4,7,10\} \\
2 & \{2,5,8,11\} \\
3 & \{3,6,9,12\} \\
4 & \{1,2,3,4,5,6\} \\
5 & \{4,5,7,8\} \\
6 & \{8,9,11,12\} \\
7 & \{10,11\}
\end{array}
$
&
$
\begin{array}{c|cc}
i & |S_i| & w_i\\
\hline
1 & 4     & 6  \\
2 & 4     & 10 \\
3 & 4     & 8  \\
4 & 6     & 6  \\
5 & 4     & 5  \\
6 & 4     & 9  \\
7 & 2     & 4
\end{array}
$
\end{tabular}
\end{center}

\noindent\textbf{Init:}
\begin{itemize}
\item $I = \{\}$
\item $\widehat{S}_j = S_j, \forall j$
\end{itemize}

\noindent\textbf{Itération 1:}
$$
\begin{array}{c|ccccccc}
j               & 1 & 2  & 3 & 4 & 5 & 6 & 7\\
\hline
w_j             & 6 & 10 & 8 & 6 & 5 & 9 & 4\\
|\widehat{S}_j| & 4 & 4  & 4 & 6 & 4 & 4 & 2
\end{array}
$$

\begin{itemize}
\item $l = 4$
\item $I = \{4\}$
\end{itemize}

$$
\begin{array}{c|c}
i & \widehat{S}_i\\
\hline
1 & \{7,10\} \\
2 & \{8,11\} \\
3 & \{9,12\} \\
4 & \{\} \\
5 & \{7,8\} \\
6 & \{8,9,11,12\} \\
7 & \{10,11\}
\end{array}
$$

\noindent\textbf{Itération 2:}
$$
\begin{array}{c|ccccccc}
j               & 1 & 2  & 3 & 4 & 5 & 6 & 7\\
\hline
w_j             & 6 & 10 & 8 & / & 5 & 9 & 4\\
|\widehat{S}_j| & 2 & 2  & 2 & / & 2 & 4 & 2
\end{array}
$$

\begin{itemize}
\item $l = 7$
\item $I = \{4, 7\}$
\end{itemize}

$$
\begin{array}{c|c}
i & \widehat{S}_i\\
\hline
1 & \{7\} \\
2 & \{8\} \\
3 & \{9,12\} \\
4 & \{\} \\
5 & \{7,8\} \\
6 & \{8,9,12\} \\
7 & \{\}
\end{array}
$$

\noindent\textbf{Itération 3:}
$$
\begin{array}{c|ccccccc}
j               & 1 & 2  & 3 & 4 & 5 & 6 & 7\\
\hline
w_j             & 6 & 10 & 8 & / & 5 & 9 & /\\
|\widehat{S}_j| & 1 & 1  & 2 & / & 2 & 3 & /
\end{array}
$$

\begin{itemize}
\item $l = 5$
\item $I = \{4, 7, 5\}$
\end{itemize}

$$
\begin{array}{c|c}
i & \widehat{S}_i\\
\hline
1 & \{\} \\
2 & \{\} \\
3 & \{9,12\} \\
4 & \{\} \\
5 & \{\} \\
6 & \{9,12\} \\
7 & \{\}
\end{array}
$$

\noindent\textbf{Itération 4:}
$$
\begin{array}{c|ccccccc}
j               & 1 & 2 & 3 & 4 & 5 & 6 & 7\\
\hline
w_j             & / & / & 8 & / & / & 9 & /\\
|\widehat{S}_j| & / & / & 2 & / & / & 2 & /
\end{array}
$$

\begin{itemize}
\item $l = 3$
\item $I = \{4, 7, 5, 3\}$
\end{itemize}

$$
\begin{array}{c|c}
i & \widehat{S}_i\\
\hline
1 & \{\} \\
2 & \{\} \\
3 & \{\} \\
4 & \{\} \\
5 & \{\} \\
6 & \{\} \\
7 & \{\}
\end{array}
$$

\noindent $\Rightarrow$ La solution est donc $(0, 0, 1, 1, 1, 0, 1)$.
$\text{APP} = 23$ et $\text{OPT} = 21$ (avec $S_1, S_4 \text{ et } S_6$).

\end{solution}

\begin{exercice}
Calculez le paramètre $g$ pour la méthode précédente et vérifiez son facteur
d'approximation.
\end{exercice}

\begin{solution}
Pour \algo{GREEDY-SC}, $g = \max\limits_{j}{|S_j}$. Ici, $g=6$.\\

\algo{GREEDY-SC} a unfacteur d'approx $H_g$, le $g^\text{ième}$ nombre
harmonique (c.f. notes de cours).
$$
H_g = \sum\limits_{i=1}^{g}{\frac{1}{i}} = 2.45
$$
\end{solution}

\end{document}
