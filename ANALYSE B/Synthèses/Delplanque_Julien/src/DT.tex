\documentclass[a4paper,11pt]{report}
 
\usepackage[utf8]{inputenc}  
\usepackage[T1]{fontenc}
\usepackage[francais]{babel}
\usepackage{amsfonts}
\usepackage{amsmath}
\usepackage{listings}
\usepackage{fullpage}
\usepackage{fancyhdr}
\pagestyle{fancy}

\renewcommand{\thesection}{}
\renewcommand{\thesubsection}{}

\renewcommand{\headrulewidth}{0pt}
\fancyhead[C]{} 
\fancyhead[L]{}
\fancyhead[R]{}

\renewcommand{\footrulewidth}{1pt}
\fancyfoot[C]{\textbf{\thepage}} 
\fancyfoot[L]{Delplanque Julien}
\fancyfoot[R]{2013-2014}

\begin{document}
\renewcommand{\labelitemi}{$\cdot$}
\begin{Large}\begin{center} 
   \underline{\textbf{Analyse I (Partie B): Développements de Taylor et séries}} 
\end{center}\end{Large}

\subsection{DT d'ordre $n$ en $0$ à connaitre par coeur (pour rapidité)}
\begin{itemize}
	\item $\frac{1}{1+x}=1-x+x^2-x^3+...+(-1)^n x^n+o(x^n)$\\
	\item $\frac{1}{1-x}=1+x+x^2+x^3+...+x^n+o(x^n)$\\
	\item $e^x=1+x+\frac{x^2}{2}+\frac{x^3}{6}+...+\frac{x^n}{n!}+o(x^n)$\\
	\item $cos(x)=1-\frac{x^2}{2}+\frac{x^4}{24}+...+(-1)^n\frac{x^{2n}}{(2n)!}+o(x^{2n+1})$\\
	\item $sin(x)=x-\frac{x^3}{6}+\frac{x^5}{120}+...+(-1)^n\frac{x^{2n+1}}{(2n+1)!}+o(x^{2n+2})$\\
	\item $ch(x)=1+\frac{x^2}{2}+\frac{x^4}{24}+...+\frac{x^{2n}}{(2n)!}+o(x^{2n+1})$\\
	\item $sh(x)=x+\frac{x^3}{6}+\frac{x^5}{120}+...+\frac{x^{2n+1}}{(2n+1)!}+o(x^{2n+2})$\\
\end{itemize}

\subsection{Séries: critères de convergence}
\begin{itemize}
	\item si $\sum\limits_{n=0}^{+\infty}{X_n}$ converge, alors $X_n \rightarrow 0$\\
	\item si $\sum\limits_{n=0}^{+\infty}{|X_n|}$ converge, alors si $\sum\limits_{n=0}^{+\infty}{X_n}$ converge.\\
	\item Soient $(X_n)_{n\in\mathbb{N}}$ et $(Y_n)_{n\in\mathbb{N}}$ tels que $\forall n \in \mathbb{N}, |X_n| \le Y_n$,\\
	Si $\sum\limits_{n=0}^{+\infty}{Y_n}$ converge, alors $\sum\limits_{n=0}^{+\infty}{X_n}$ converge.\\
	\item Soit $x \in \mathbb{R}$,\\
	Si $|x|<1$, alors $\sum\limits_{n=0}^{+\infty}{x^n}$ converge et vaut $\frac{1}{1-x}$.\\
	\item Critère d'Alembert (valeure absolue).\\
	\item Critère de Cauchy (racine).\\
	\item Si $\alpha > 1$, alors $\sum\limits_{n=0}^{+\infty}{\frac{1}{n^\alpha}}$ converge.\\
	\item Une suite est de Cauchy ssi $\forall \varepsilon>0, \exists n_0 \in \mathbb{N}, \forall n \ge m \ge n_0, |X_m-X_n| \le \varepsilon$.
\end{itemize}

\end{document}