\documentclass[a4paper,11pt]{report}
 
\usepackage[utf8]{inputenc}  
\usepackage[T1]{fontenc}
\usepackage[francais]{babel}
\usepackage{amsfonts}
\usepackage{amsmath}
\usepackage{listings}
\usepackage{fullpage}
\usepackage{fancyhdr}
\pagestyle{fancy}

\renewcommand{\headrulewidth}{0pt}
\fancyhead[C]{} 
\fancyhead[L]{}
\fancyhead[R]{}

\renewcommand{\footrulewidth}{1pt}
\fancyfoot[C]{\textbf{\thepage}} 
\fancyfoot[L]{Delplanque Julien}
\fancyfoot[R]{2013-2014}

\begin{document}
\begin{center} 
   \underline{\textbf{Analyse I (Partie B): Plan du cours}} 
\end{center} 
\renewcommand{\labelitemi}{$\bullet$}
\renewcommand{\labelitemii}{$\cdot$}
\renewcommand{\labelitemiii}{$\diamond$}
\renewcommand{\labelitemiv}{$\ast$}
\begin{itemize}
	\item Limite et continuité de fonctions d'une variable réelle.
	\begin{itemize}
		\item Adhérence du domaine
		\item Limite de fonction en terme de suite
		\item Unicité de la limite d'une fonction
		\item Règles de calcul
		\item Convergence dominée
		\item Recouvrement exhaustif
		\item Limite de fonction en $\varepsilon-\delta$
		\item Continuité + propriétés
		\item Théorème des valeurs intermédiaires
		\item Théorème des valeurs intermédiaires généralisé
		\item Algorithme pour estimer $\xi$
		\item Maximum et minimum d'une fonction
		\item Théorème des bornes atteintes + interprétation géométrique
		\item Interval compact
	\end{itemize}
	\item Dérivée de fonction d'une variable réelle
	\begin{itemize}
		\item Interprétation géométrique
		\item Dérivabilité d'une fonction + propriété (dérivable $\rightarrow$ continue)
		\item Régles de calcul
		\item Théorème de la moyenne + propriété de croissance
		\item Théorème de Rolle + équivalence avec le théorème de la moyenne
		\item Minimum/Maximum local + propriété
		\item "Petit o" + propriété
	\end{itemize}
	\item Développement de Taylor (DT) et séries
	\begin{itemize}
		\item "Petit o"
		\item DT
		\item Unicité du DT
		\item Calcul du DT
		\item L'ensemble $C^k$ où $k\in\mathbb{N}$ et $C^\infty$
		\item Formule du reste + lien avec le théorème de la moyenne
		\item Rappel du Binôme de Newton
		\item Séries
		\item Convergence d'une série
		\item Convergence absolue + propriété (convergence absolue $\rightarrow$ convergence)
		\item Critère du quotient
		\item Critère de la racine
	\end{itemize}
	\item Les équation différentielles ordinaires (EDO)
	\begin{itemize}
		\item Équation différentielles ordinaires
		\item Solution d'une EDO
		\item Théorème: Condition d'une unique solution
		\item EDO linéaire (affine)
		\item Principe de superposition
		\item EDO homogène
		\item Résolution de $Lu=0$
		\item Solution réelle
		\item EDO inhomogène
		\item Théorème: Forme de la solution particulière
	\end{itemize}
\end{itemize}
\end{document}