\documentclass[a4paper,11pt]{report}
 
\usepackage[utf8]{inputenc}  
\usepackage[T1]{fontenc}
\usepackage[francais]{babel}
\usepackage{amsfonts}
\usepackage{amsmath}
\usepackage{listings}
\usepackage{fullpage}
\usepackage[hidelinks]{hyperref}
\usepackage{fancyhdr}
\pagestyle{fancy}

\renewcommand{\thesection}{}
\renewcommand{\thesubsection}{}

\renewcommand{\headrulewidth}{0pt}
\fancyhead[C]{} 
\fancyhead[L]{}
\fancyhead[R]{}

\renewcommand{\footrulewidth}{1pt}
\fancyfoot[C]{\textbf{\thepage}} 
\fancyfoot[L]{Delplanque Julien}
\fancyfoot[R]{2013-2014}

\begin{document}
\renewcommand{\labelitemi}{$\cdot$}
\renewcommand{\labelitemii}{$-$}
\begin{Large}\begin{center} 
   \underline{\textbf{Analyse I (Partie B): $\forall n \in \mathbb{N}, \sqrt[n\,]{x}$ est continue sur son domaine.}} 
\end{center}\end{Large}

Tenons pour vrai que:
\begin{center}
(i) $x-a = (\sqrt[n\,]{x}-\sqrt[n\,]{a})\sum\limits_{k=0}^{n-1}{\sqrt[n\,]{x^{n-1-k}}\sqrt[n\,]{a^k}}$\\
(ii) $\forall n \in \mathbb{N}, n$ pair $\rightarrow Dom(\sqrt[n\,]{x}) = \mathbb{R}^+$\\
(iii) $\forall n \in \mathbb{N}, n$ impair $\rightarrow Dom(\sqrt[n\,]{x}) = \mathbb{R}$\\
\end{center}

\textbf{Montrons que $\forall n \in \mathbb{N}, \sqrt[n\,]{x}$ est continue sur son domaine.}
\begin{center}
i.e $\forall n \in \mathbb{N}, \forall a \in Dom(\sqrt[n\,]{x}), \forall \varepsilon > 0, \exists \delta > 0, \forall x \in Dom(\sqrt[n\,]{x}), |x-a| \le \delta \rightarrow |\sqrt[n\,]{x}-\sqrt[n\,]{a}| \le \varepsilon$
\end{center}
Preuve:\\
Soit $n \in \mathbb{N}$,
\begin{itemize}
	\item \textbf{Si $n$ est pair}, alors le domaine de $\sqrt[n\,]{x}$ est $\mathbb{R}^+$. (ii)\\
	Soit $a \in \mathbb{R}^+$,\\
	Soit $\varepsilon > 0$,\\
	Prenons $\delta = \varepsilon \sqrt[n\,]{a^{n-1}} > 0$,\\
	Soit $x \in \mathbb{R}^+$ tel que $|x-a| \le \delta$,\\
	On a:\\
	\begin{center}
		$|\sqrt[n\,]{x}-\sqrt[n\,]{a}| = |\frac{(\sqrt[n\,]{x}-\sqrt[n\,]{a})\sum\limits_{k=0}^{n-1}{(\sqrt[n\,]{x^{n-1-k}}\sqrt[n\,]{a^k})}}{\sum\limits_{k=0}^{n-1}{(\sqrt[n\,]{x^{n-1-k}}\sqrt[n\,]{a^k})}}|$\\
		
		i.e $|\sqrt[n\,]{x}-\sqrt[n\,]{a}| = \frac{|x-a|}{|\sum\limits_{k=0}^{n-1}{(\sqrt[n\,]{x^{n-1-k}}\sqrt[n\,]{a^k})}|}$ (i) et par une prop. des $|.|$\\
		
		i.e $|\sqrt[n\,]{x}-\sqrt[n\,]{a}| \le \frac{\delta}{|\sum\limits_{k=0}^{n-1}{(\sqrt[n\,]{x^{n-1-k}}\sqrt[n\,]{a^k})}|}$ par hyp.\\
		
		i.e $|\sqrt[n\,]{x}-\sqrt[n\,]{a}| \le \frac{\delta}{|\sqrt[n\,]{x^{n-1}}+\sqrt[n\,]{x^{n-2}}\sqrt[n\,]{a}+...+\sqrt[n\,]{x}\sqrt[n\,]{a^{n-2}}+\sqrt[n\,]{a^{n-1}}|} \stackrel{(\alpha)}{\le}  \frac{\delta}{\sqrt[n\,]{a^{n-1}}} \le \varepsilon$\\		
	\end{center}
	$(\alpha)$: $|\sqrt[n\,]{x^{n-1}}+\sqrt[n\,]{x^{n-2}}\sqrt[n\,]{a}+...+\sqrt[n\,]{x}\sqrt[n\,]{a^{n-2}}+\sqrt[n\,]{a^{n-1}}|$\\
	= $\sqrt[n\,]{x^{n-1}}+\sqrt[n\,]{x^{n-2}}\sqrt[n\,]{a}+...+\sqrt[n\,]{x}\sqrt[n\,]{a^{n-2}}+\sqrt[n\,]{a^{n-1}}$\\ Car $a>0$ et $x>0$\\
	Par transitivité,\\
	\begin{center}
		$|\sqrt[n\,]{x}-\sqrt[n\,]{a}| \le \varepsilon$\\
	\end{center}
	Cqfd\\
	
	\item \textbf{Si $n$ est impair}, alors le domaine de $\sqrt[n\,]{x}$ est $\mathbb{R}$. (iii)\\
	Soit $a \in \mathbb{R}$,\\
	\begin{itemize}
		\item \textbf{Si $a > 0$},\\
		Soit $\varepsilon > 0$,\\
		Prenons $\delta = \min{(\frac{a}{2},\varepsilon \sqrt[n\,]{a^{n-1}})}>0$ ,\\
		Soit $x \in \mathbb{R}$ tel que $|x-a| \le \delta$ i.e $a-\delta \le x \le a+\delta$ donc en prenant $\delta = \frac{a}{2} > 0$, on a $a-\frac{a}{2} \le x \le a+\frac{a}{2}$ i.e $0 < \frac{a}{2} \le x \le \frac{3a}{2}$. Donc par transitivé, $0 < x$.\\
		On a:\\
		\begin{center}
		$|\sqrt[n\,]{x}-\sqrt[n\,]{a}| = |\frac{(\sqrt[n\,]{x}-\sqrt[n\,]{a})\sum\limits_{k=0}^{n-1}{(\sqrt[n\,]{x^{n-1-k}}\sqrt[n\,]{a^k})}}{\sum\limits_{k=0}^{n-1}{(\sqrt[n\,]{x^{n-1-k}}\sqrt[n\,]{a^k})}}|$\\
		
		i.e $|\sqrt[n\,]{x}-\sqrt[n\,]{a}| = \frac{|x-a|}{|\sum\limits_{k=0}^{n-1}{(\sqrt[n\,]{x^{n-1-k}}\sqrt[n\,]{a^k})}|}$ (i) et par une prop des $|.|$\\
		
		i.e $|\sqrt[n\,]{x}-\sqrt[n\,]{a}| \le \frac{\delta}{|\sum\limits_{k=0}^{n-1}{(\sqrt[n\,]{x^{n-1-k}}\sqrt[n\,]{a^k})}|}$ par hyp.\\
		
		i.e $|\sqrt[n\,]{x}-\sqrt[n\,]{a}| \le \frac{\delta}{|\sqrt[n\,]{x^{n-1}}+\sqrt[n\,]{x^{n-2}}\sqrt[n\,]{a}+...+\sqrt[n\,]{x}\sqrt[n\,]{a^{n-2}}+\sqrt[n\,]{a^{n-1}}|} \le  \frac{\delta}{\sqrt[n\,]{a^{n-1}}} \le \varepsilon$\\
		\end{center}
		Par transitivité,\\
		\begin{center}
			$|\sqrt[n\,]{x}-\sqrt[n\,]{a}| \le \varepsilon$\\
		\end{center}
		Cqfd\\
		
		\item \textbf{Si $a = 0$},\\
		Soit $\varepsilon > 0$,\\
		Prenons $\delta = \varepsilon^n>0$ ,\\
		Soit $x \in \mathbb{R}$ tel que $|x-a| \le \delta$ i.e $|x| \le \delta$ i.e $-\delta \le x \le \delta$ donc en prenant $\delta = \varepsilon^n$, on a $-\varepsilon^n \le x \le \varepsilon^n$.\\ 
		Comme $\forall x \in \mathbb{R},\forall m \in \mathbb{N}, (-x)^{2m+1} = -x^{2m+1}$, on a $(-\varepsilon)^n \le x \le \varepsilon^n$ i.e $-\varepsilon \le \sqrt[n\,]{x} \le \varepsilon$ cette dernière inégalité est équivalente à $|\sqrt[n\,]{x}| \le \varepsilon$.\\
		Cqfd\\
		
		\item \textbf{Si $a < 0$},\\
		Soit $\varepsilon > 0$,\\
		Prenons $\delta = \min{(\frac{|a|}{2},\varepsilon \sqrt[n\,]{a^{n-1}})} > 0$ ,\\
		Soit $x \in \mathbb{R}$ tel que $|x-a| \le \delta$ i.e $|x| \le \delta$ i.e $a-\delta \le x \le a+\delta$ donc en prenant $\delta = \frac{|a|}{2} = \frac{-a}{2}> 0$ car $a < 0$, on a $\frac{3a}{2} \le x \le \frac{a}{2}<0$.\\ Donc par transitivité, $x<0$.\\
		On a:\\
		\begin{center}
		$|\sqrt[n\,]{x}-\sqrt[n\,]{a}| = |\frac{(\sqrt[n\,]{x}-\sqrt[n\,]{a})\sum\limits_{k=0}^{n-1}{(\sqrt[n\,]{x^{n-1-k}}\sqrt[n\,]{a^k})}}{\sum\limits_{k=0}^{n-1}{(\sqrt[n\,]{x^{n-1-k}}\sqrt[n\,]{a^k})}}|$\\
		
		i.e $|\sqrt[n\,]{x}-\sqrt[n\,]{a}| = \frac{|x-a|}{|\sum\limits_{k=0}^{n-1}{(\sqrt[n\,]{x^{n-1-k}}\sqrt[n\,]{a^k})}|}$ (i) et par unr prop. des $|.|$\\
		
		i.e $|\sqrt[n\,]{x}-\sqrt[n\,]{a}| \le \frac{\delta}{|\sum\limits_{k=0}^{n-1}{(\sqrt[n\,]{x^{n-1-k}}\sqrt[n\,]{a^k})}|}$ par hyp.\\
		
		i.e $|\sqrt[n\,]{x}-\sqrt[n\,]{a}| \le \frac{\delta}{|\sqrt[n\,]{x^{n-1}}+\sqrt[n\,]{x^{n-2}}\sqrt[n\,]{a}+...+\sqrt[n\,]{x}\sqrt[n\,]{a^{n-2}}+\sqrt[n\,]{a^{n-1}}|} \le \frac{\delta}{\sqrt[n\,]{a^{n-1}}} \le \varepsilon$\\
		\end{center}
		Par transitivité,\\
		\begin{center}
			$|\sqrt[n\,]{x}-\sqrt[n\,]{a}| \le \varepsilon$\\
		\end{center}
		Cqfd\\
	\end{itemize}
\end{itemize}
\textbf{Conclusion}: On a bien prouvé que $\forall n \in \mathbb{N}, \sqrt[n\,]{x}$ est continue sur son domaine.
\end{document}